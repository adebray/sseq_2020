\subsection{Fibration exercises}

\begin{xca}
	Show that if $f: X \to B$ is a Serre fibration with $B$ path-connected, then the fibers over any two points are homotopy equivalent.  That is, $f^{-1}(b_1) \simeq f^{-1}(b_2)$.
\end{xca}

\begin{xca}
	Show that a Serre fibration $F \to E \to B$ induces a long exact sequence of homotopy groups $$\dots \to \pi_n(F) \to \pi_n(E) \to \pi_n(B) \to \pi_{n-1}(F) \to \dots \to \pi_0(E)$$
\end{xca}

\begin{xca}
	Given a short exact sequence of groups $H \to G \to G/H$, show that there is a Serre fibration of classifying spaces $BH \to BG \to BG/H$
\end{xca}

\begin{xca}
	Show that the notion of Serre fibration is strictly weaker than the notion of a Hurewicz fibration.
\end{xca}

\begin{xca}
	Show that the fibration $G \to EG \to BG$ can be obtained from the path space fibration $\Omega BG \to BG^I \to BG$.
\end{xca}

\begin{xca}
	Given a universal cover $\Tilde{X} \to X$ with $\pi_1(X) = G$, show that we have a fibration $\tilde{X} \to X \to BG$.
	
	In general, If $G$ acts on a space $X$ such that the quotient map $X \to X/G$ is a covering space, show that we have a fibration $X \to X/G \to BG$.
\end{xca}

\subsection{Spectral Sequence Computations}

\begin{xca}
	Given a universal cover $\Tilde{X} \to X$ with $\pi_1(X) = G$, use the Serre spectral sequence to show that there is an isomorphism $H^*(X; \Q) \to (H^*(\Tilde{X}; \Q))^G$.
	
	How can this statement be generalized? For example, how necessary is the coefficient ring $\Q$?
\end{xca}

\begin{xca}
	Show that if $F \to E \to B$ is a Serre fibration with $\pi_1(B)$ acting trivially, and we take coefficients $A = k$ for some field $k$, then the Serre spectral sequence takes the form 
	$$E_2^{s,t} = H^p(B;k) \otimes H^q(F;k) \Rightarrow H^{p+q}(E;k)$$
\end{xca}

\begin{xca}
	Play around with the Serre spectral sequence for the fibration $S^n \to D^n \to S^{n+1}$.
\end{xca}

\begin{xca}
	Play around with the Serre spectral sequence for the Hopf fibration $S^1 \to S^3 \to S^2$.
\end{xca}

\begin{xca}
	Play around with the Serre spectral sequence for the fibration $U(n-1) \to U(n) \to S^{2n-1}$.
\end{xca}

\begin{xca}
	Play around with the Serre spectral sequence for the fibration $SO(n) \to SO(n+1) \to S^n$.
\end{xca}

\begin{xca}
	Let $V_2(\R^{n+1})$ be the space of orthogonal pairs of vectors in $\R^{n+1}$.
	
	
	\begin{enumerate}
		\item Show we have a Serre fibration $S^2 \to V_2(\R^{n+1}) \to S^n$
		\item Compute $H^*(V_2(\R^{n+1}))$.
	\end{enumerate}
	
\end{xca}

\begin{xca}
	Compute the cup product structure on $H^*(\Omega S^n)$ using the path space fibration $\Omega S^n \to (S^n)^I \to S^n$.
\end{xca}

\begin{xca}
	Compare the spectral sequence for the fibration $S^2 \to S^2 \times S^2 \to S^2$ with the fibration $S^2 \to X \to S^2$, where $X$ is built by taking two mapping cylinders of the Hopf map $S^3 \to S^2$, and gluing them together along the identity on $S^3$.
	
	Show that $H^*(S^2 \times S^2)$ and $H^*(X)$ have different ring structures.
	
\end{xca}

\begin{xca}
	
	Prove (recover) the Gysin sequence.
	
	\begin{theorem*}[The Gysin Sequence]
		
		Let $S^n \to E \to B$ be a Serre fibration with $B$ simply connected and $n \geq 1$.  There exists a long exact sequence $$ \cdots \to H^k(B) \to H^k(X) \to H^{k-n}(B) \to H^{k+1}(B) \to \cdots$$
		
	\end{theorem*}
	
	
\end{xca}

\begin{xca}
	Prove (recover) the Wang sequence.
	
	\begin{theorem*}[The Wang Sequence]
		
		Let $F \to X \to S^n$ be a Serre fibration with $B$ simply connected and $n \geq 1$.  There exists a long exact sequence $$ \cdots \to H^{k-1}(F) \to H^{k-n}(F) \to H^{k}(X) \to H^{k}(F) \to \cdots$$
		
	\end{theorem*}
	
\end{xca}


\begin{xca}
	
	
	Prove (recover) this Hurewicz isomorphism using the path fibration $$\Omega(X) \to PX \to X$$
	\begin{theorem*}[Hurewicz]
		
		Let $X$ be an $(n-1)$-connected space, with $n \geq 2$.  Then $\tilde{H}_i(X) = 0$ for $i \leq n-1$, and we have the Hurewicz isomorphism $$\pi_n(X) \cong H_n(X)$$
		
	\end{theorem*}
	
\end{xca}

\begin{xca}
	
	Prove (recover) the Leray-Hirsch Theorem.
	
	\begin{theorem*}[Leray-Hirsch]
		
		Let $F \to E \to B$ be a fiber bundle such that $F$ is of finite type.  That is, that $H^p(F;\Q)$ is finite dimensional for all $p$.
		
		Furthermore, assume that the inclusion $i: F \to E$ induces a surjection $$i^*: H^*(E;\Q) \to H^*(F;\Q)$$
		
		Then we have an isomorphism of $H^*(B;\Q)$-modules $$H^*(F;\Q) \otimes_\Q H^*(B;\Q) \cong H^*(E; \Q)$$
		
	\end{theorem*}
	
\end{xca}

\begin{xca}
	How can the Leray-Hirsch theorem above be generalized? In particular, how necessary is the coefficient ring $\Q$?
\end{xca}


\subsection{For your enlightenment}


\begin{xca}
	Show that there is a relationship between the bigraded chain complex $$\cdots \to H^*(E_{s-1},E_{s})\xrightarrow{d}  H^*(E_s,E_{s-1}) \xrightarrow{d} H^*(E_{s+1},E_{s}) \to \cdots $$
	and $H^*(B)$ and $H^*(F)$.
	
	Namely, that there is an isomorphism $$E_1^{s,t} \cong C^s(B;H^{t}(F))$$
	where $C^*(B;H^{t}(F))$ is the cellular cochain complex for $B$ with coefficients in $H^{t}(F)$.
\end{xca}


\begin{xca}
	What was special about the Serre filtration on $X$?  Can you construct exact couples using a different filtration?  Can you construct a spectral sequence using a different filtration?
\end{xca}

\begin{xca}
	What was special about using cohomology?  Can you construct a homological Serre spectral sequence?  
	
	Can you construct a spectral sequence using a generalized cohomology theory?
\end{xca}