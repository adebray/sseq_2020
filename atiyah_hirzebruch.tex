\TODO: this is taken directly from my notes when I gave this lecture three years ago, and will need to be modified.
I'll indicate what needs to be changed where it needs to be changed.

Today, I'm going to talk about the Atiyah-Hirzebruch spectral sequence. This spectral sequence computes generalized
(co)homology of a space or spectrum, with input data its ordinary homology.

Let $D$ be a spectrum and $X$ be a CW complex. The \term{homological Atiyah-Hirzebruch spectral sequence} is a
spectral sequence with signature
\begin{equation}
	E^2_{p,q} \coloneqq H_p(X; D_q(\pt)) \Longrightarrow D_{p+q}(X).
\end{equation}
The \term{cohomological Atiyah-Hirzebruch spectral sequence} is a spectral sequence with signature
\begin{equation}
\label{coh_AHSS}
	E_2^{p,q} \coloneqq H^p(X; D^q(\pt)) \Longrightarrow D^{p+q}(X).
\end{equation}
(Briefly recall what this means. Briefly discuss convergence, and where it can go wrong.)

If $D$ is a ring spectrum, \eqref{coh_AHSS} has a multiplicative structure.

\TODO: this would go in day 1, if we want to include it at all. Feel free to port it over there; if not, comment it
out.
\subsection*{Convergence}
Sometimes you're reading a book and it feels like it goes on forever. It's nice when spectral sequences don't do
that. As an example, we'll look at a \term{first-quadrant spectral sequence}, one where $E^{p,q}_2 = 0$ when $p <
0$ or $q < 0$. In this setup, if you pick any $(p,q)$, then after finitely many pages, the differentials are so
long that they leave the first quadrant, so you get a sequence $0\to E_{p,q}^r\to 0$, and therefore when you take
homology, nothing changes. Thus it makes sense to say what the end of the spectral sequence is.
\begin{defn}
Whenever it makes sense, we'll define the \term{$E_\infty$-page} of the spectral sequence to be $E_\infty^{p,q} =
E_{p,q}^r$ for $r\gg 0$. One says $E_r^{p,q}$ \term{converges} or \term{abuts} to $E_\infty^{p,q}$.
\end{defn}
Typically this is something interesting we want to calculate.
\begin{defn}
Let $A_\bullet$ be a graded abelian group together with an exhaustive filtration $\set{F_p}$.
\begin{itemize}
	\item The \term{associated graded} of the filtration $\set{F_i}$ is
	\[(\mathit{gr} A)_{p,q}\coloneqq F_pA_{p+q}/F_{p-1}A_{p+q}.\]
	\item A spectral sequence $E_r^{p,q}$ \term{converges (weakly)} to $A_\bullet$, written
	\[E_r^{p,q}\Longrightarrow A_\bullet,\]
	if it has an $E_\infty$ page and the $E_\infty$ page is the associated graded of $A_\bullet$.
\end{itemize}
\end{defn}
\begin{rem}
There is a notion of \term{conditional convergence}, due to Boardman, which essentially means ``not always weakly
convergent, but converges under hypotheses often met in practice.'' Unfortunately, defining this precisely would be
a huge digression.
\end{rem}


% 3. generalized cohomology theories
%\subsection*{Generalized cohomology theories}
%The Atiyah-Hirzebruch spectral sequence is used to compute things which behave like homology or cohomology, but
%are slightly different: they satisfy all of the Eilenberg-Steenrod axioms except for the dimension axiom. These
%generalized cohomology theories have been a huge area of focus in algebraic topology in the last half century.
%\begin{defn}
%A \term{generalized cohomology theory} (also \term{extraordinary cohomology theory}) is a collection of functors
%$h^n\colon\Top_*\to\Ab$ such that:
%\begin{itemize}
%	\item Given a map $f\colon A\to X$, let $X/A$ denote its cofiber. There is a natural transformation
%	$\delta\colon h^n(X/A)\to h^{n+1}(A)$ such that the following sequence is long exact:
%	\[\xymatrix{
%		\dotsb \ar[r] &h^n(A)\ar[r]^{h^n(f)} & h^n(X)\ar[r] & h^n(X/A)\ar[r]^\delta & h^{n+1}(A)\ar[r] &\dotsb
%	}\]
%	\item $h^n$ takes wedge sums to direct sums: if $X = \bigvee_i X_i$, then the natural map
%	\[\bigoplus h^n(X_i)\longrightarrow h^n(X)\]
%	is an isomorphism.
%\end{itemize}
%\end{defn}
%The dual notion of a \term{generalized homology theory} is the same, except the differentials go in the other
%direction. This defines a reduced homology theory, i.e.\ one for spaces with basepoints.

\TODO: these are some examples of spectra. Probably I won't delve into this level of
detail, just introduce $\KO$, $\KU$, $\ko$, $\ku$, and $\mathit{MSO}$ and say what their coefficient rings are, as
well a little bit about what they mean. Some of this may have already been covered in Tuesday's lecture.
\begin{exm}[$K$-theory]
Let $X$ be a compact Hausdorff space. Then, the set of isomorphism classes of complex vector bundles on $X$ is a
semiring, so we can take its group completion and obtain a ring $K^0(X)$.

The following theorem is foundational and beautiful.
\begin{thm}[Bott periodicity]
$K^0(\Sigma^2 X)\cong K^0(X)$.
\end{thm}
This allows us to promote $K^*$ into a \term{$2$-periodic} generalized cohomology theory $K^*$, called
\term{complex $K$-theory}, by setting $K^{2n}(X) = K^0(X)$ and $K^{2n+1}(X) = K^0(\Sigma X)$.\footnote{Extending
from compact Hausdorff spaces to all of $\mathsf{Top}$ is possible, but then one loses the vector-bundle-theoretic
description.}

Like cohomology, $K$-theory is \term{multiplicative}, i.e.\ it spits out $\Z$-graded rings. However, $K^i(X)$ is
often nonzero for negative $i$.
\begin{ex}
For example, show that as graded abelian groups, $K^*(\pt) = \Z[t,t^{-1}]$, where $\abs t = 2$.
\end{ex}

$K$-theory admits a few variants.
\begin{itemize}
	\item If you use real vector bundles instead of complex vector bundles, everything still works, but Bott
	periodicity is $8$-fold periodic. Thus we obtain a periodic, multiplicative cohomology theory called \term{real
	$K$-theory}, denoted $\KO^*(X)$. Its value on a point is encoded in the \term{Bott song}.
	\item Sometimes it will be simpler to consider a smaller variant where we only keep the negative-degree
	elements. This is called \term{connective $K$-theory}, and is denoted $\mathit{ku}^*$ (for complex $K$-theory)
	or $\mathit{ko}^*$ (for real $K$-theory). These are also multiplicative. \qedhere
\end{itemize}
\end{exm}
\begin{exm}[Bordism]
Let $X$ be a space and define $\Omega_n^\O(X)$ to be the set of equivalence classes of maps of $n$-manifolds $M\to
X$, where $[f_0\colon M\to X]\sim [f_1\colon N\to X]$ if there's a cobordism $Y\colon M\to N$ and a map $F\colon
Y\to X$ extending $f_0$ and $f_1$. This is an abelian group under disjoint union, and the collection
$\set{\Omega_n^\O}$ defines a generalized homology theory called \term{unoriented bordism}.\footnote{The
corresponding cohomology theory is called \term{cobordism}.}

The following theorem was the beginning of differential topology.
\begin{thm}[Thom]
As graded abelian groups, $\Omega_n^\O(\pt)\cong\F_2[x_2,x_4,x_5,x_6,\dotsc] = \F_2[x_i\mid i\ne 2^j-1]$. Moreover,
$\Omega_*^\O$ is a direct sum of (suspended) ordinary cohomology theories.
\end{thm}
There's a lot of variations, based on whatever flavors of manifolds you consider. Using oriented manifolds produces
\term{oriented bordism} $\Omega_*^\SO$, spin manifolds produce \term{spin bordism} $\Omega_*^\Spin$, and so forth.
These are not direct sums of ordinary cohomology theories in general.
\end{exm}

%. 4. then: CW filtration => AHSS (homological and cohomological)
%\subsection{The definition}
%Recall that if $X$ is a CW complex, it has a \term{CW filtration} in which $X_n$ is the \term{$n$-skeleton}, the
%union of all cells of dimension $\le n$. Then, $X_n/X_{n-1}$ is a wedge of $n$-spheres indexed by the $n$-cells of
%$X$. Using this formalism we can define some spectral sequences.
%\begin{comp}{defn}{itemize}
%	\item Let $E_*$ be a generalized homology theory and $X$ be a CW complex. Then, the CW filtration on $X$
%	induces a spectral sequence of homological type that strongly converges, called the \term{Atiyah-Hirzebruch
%	spectral sequence}:
%		\[E_{p,q}^2 = H_p(X; E_q(\pt))\Longrightarrow E_{p+q}(X).\]
%	\item Let $E^*$ be a generalized cohomology theory and $X$ be a CW complex. Then, the CW filtration on $X$
%	induces a spectral sequence of cohomological type that \emph{conditionally} converges, called the
%	\term{Atiyah-Hirzebruch spectral sequence}:
%		\[E^{p,q}_2 = H^p(X; E^q(\pt))\Longrightarrow E^{p+q}(X).\]
%\end{comp}



% then: calculations
% TODO add spheres

\begin{exm}
% \term{Complex $K$-theory}, denoted $K^*$ or $\KU^*$, is the generalized cohomology theory such that if $X$ is a
% compact Hausdorff space, $K^0(X)$ is the Grothendieck group of complex vector bundles on $X$. The axioms of a
% generalized cohomology theory mean that we want $K^*(X) \cong K^{*+1}(\Sigma X)$, and a computation called
% \term{Bott periodicity} shows that $K^0(\Sigma^2 X)\cong K^0(X)$; putting these together, we obtain a
% \term{$2$-periodic cohomology theory}: $K^{i+2}(X)\cong K^i(X)$ for all $i$. In particular, $K^i(X)$ may be zero
% for negative $i$.
% 
% The coefficients are
% \[K^i(\pt) = \begin{cases}
% 	\Z, &i\text{ even}\\
% 	0, &i\text{ odd.}
% \end{cases}\]
We'll use the Atiyah-Hirzebruch spectral sequence to compute $K^*(\CP^n)$. Recall that
\[H^p(\CP^k;A) = \begin{cases}
	A, &p\text{ even}\\
	0, &\text{ odd.}
\end{cases}\]
Hence
\[E_2^{p,q} = \begin{cases}
	\Z, & p,q\text{ even, } 0\le p\le 2k\\
	0, &\text{otherwise.}
\end{cases}\]
Thus all the differentials are zero! So $E_2^{p,q} \cong E_\infty^{p,q}$. Hence the $E_\infty$ page has no torsion,
and therefore $K^*(\CP^n)$ is isomorphic to its associated graded. % ?
\[K^i(\CP^n) = \begin{cases}
	\Z^{n+1}, &i\text{ even}\\
	0, &\text{otherwise.}
\end{cases}\qedhere\]

\TODO: more examples.

\TODO: define $k$-invariants, introduce Steenrod squares, and state what the first differential in the AHSS is.
Use this in examples to compute $\ku^*$ and $\ko^*$ of simple things.

%\begin{ex}
%Let $\Sigma$ be a genus-$g$ orientable closed surface. Compute $K^*(\Sigma_g)$.
%\end{ex}
%
%
%\begin{ex}
%What changes when you replace $K^*$ with $\KO^*$?
%\end{ex}



\end{exm}
