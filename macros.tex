% preamble for our lecture notes (and probably also exercises)

% add whatever you need, and I'll add what I need, and we can deal with clashes as they arise
% style decisions: I (Arun) am OK with pretty much whatever

% packages
\usepackage[utf8]{inputenc}
\usepackage[T1]{fontenc}
\usepackage{lmodern}
\usepackage[margin=1in]{geometry}
\usepackage[usenames, dvipsnames]{xcolor}
\usepackage{graphicx}
\usepackage{mathtools}
\usepackage{amssymb}
\usepackage{amsthm}

\usepackage{xparse} % in order to define shortexact, below
\usepackage{etoolbox} % for quickly defining multiple things, below
\usepackage{mathrsfs}
\usepackage[all]{xy}
\usepackage[backref=page]{hyperref}
\usepackage[all]{hypcap}
\usepackage[capitalize, noabbrev]{cleveref}

% TODO: tikz and Adams macros, and spectralsequences
\usepackage{spectralsequences}
\usepackage{adamsmacros}

\newcommand{\AdamsTower}{
	\DoUntilOutOfBounds{
		\class(\lastx, \lasty+1)
		\structline
	}
}

% Forces all XY entries to be typeset with displaymath
\everyentry={\displaystyle}
%
% Short exact sequences: write
% \shortexact[f][g]{A}{B}{C}, for:
%
% f g
% 0 -> A -> B -> C -> 0,
% (or 1 -> A -> B -> C -> 1 with \shortexact*)
% Warning! if you just want
%   0 -> A -> B -> C -> 0
% with no punctuation, you must invoke this command as
%   \shortexact{A}{B}{C}{}
% or else it will gobble the next character.
\DeclareDocumentCommand{\shortexact}{s O{} O{} mmmm}{
\IfBooleanTF{#1}{ % if star
 \xymatrix{
  1\ar[r] & #4\ar[r]^-{#2} & #5\ar[r]^-{#3} & #6\ar[r] & 1#7
 }
}{ % no star
 \xymatrix{
  0\ar[r] & #4\ar[r]^-{#2} & #5\ar[r]^-{#3} & #6\ar[r] & 0#7
 }
}}
% exactly the same, but for 0 -> A -> B -> C
% same caveat about ending w/o punctuation applies to both of these
\DeclareDocumentCommand{\leftexact}{O{} O{} mmmm}{
\xymatrix{
 0\ar[r] & #3\ar[r]^-{#1} & #4\ar[r]^-{#2} & #5 #6
}}
% ... and the same, for A -> B -> C -> 0
\DeclareDocumentCommand{\rightexact}{O{} O{} mmmm}{
\xymatrix{
 {#3}\ar[r]^-{#1} & #4\ar[r]^-{#2} & #5\ar[r] & 0#6
}}

% An xymatrix environment wrapped in gathered to ensure its equation number is vertically centered
\newcommand{\gathxy}[2][]{%
\begin{gathered}
	\xymatrix#1{#2}
\end{gathered}
}

% TODO: I (Arun) am testing a way to define lots of letterlike macros quickly, based on
% https://tex.stackexchange.com/questions/30349
% This should quickly define \C for \mathbb C, \R for mathbb R, etc., but if it doesn't work, let me know
\def\do#1{\csdef{#1}{\mathbb{#1}}}
\docsvlist{Q,R,T,F,Z,C,N,RP,CP,HP}
% a few must be done manually, I think
\renewcommand{\H}{\mathbb H}
\renewcommand{\S}{\mathbb S} % maybe we want section symbols, in which case get rid of this
\renewcommand{\P}{\mathbb P}

% now \cA for \mathcal A, \cB for \mathcal B, etc.
\def\do#1{\csdef{c#1}{\mathcal{#1}}}
\docsvlist{A,B,C,D,E,F,G,H,I,J,K,L,M,N,O,P,Q,R,S,T,U,V,W,X,Y,Z}
% and \sA for \mathscr A, \sB for \mathscr B, etc.
\def\do#1{\csdef{s#1}{\mathscr{#1}}}
\docsvlist{A,B,C,D,E,F,G,H,I,J,K,L,M,N,O,P,Q,R,S,T,U,V,W,X,Y,Z}
% mathfrak
% these are just accessed via \so, \su, ...
\def\do#1{\csdef{#1}{\mathfrak{#1}}}
\docsvlist{gl,sl,sp,so,su,spin}
% these are accessed via \fg, \ft, etc., since they're just one letter each
\def\do#1{\csdef{f#1}{\mathfrak{#1}}}
\docsvlist{g,t,b,n,u,o,U}
% mathrm -- mostly Lie groups
\def\do#1{\csdef{#1}{\mathrm{#1}}}
\docsvlist{GL,SL,Sp,SO,U,SU,Spin,Pin,PGL,PSL,Sq}
% this has to be done manually, I think
\renewcommand{\O}{\mathrm O}

% mathit -- some cohomology theories
\def\do#1{\csdef{#1}{\mathit{#1}}}
\docsvlist{KO,ko,KU,ku,TMF,Tmf,tmf,BP,MSpin}

\setcounter{tocdepth}{1}
\numberwithin{equation}{section}

% amsthm stuff
\newtheorem{thm}[equation]{Theorem}
\newtheorem{lem}[equation]{Lemma}
\newtheorem{cor}[equation]{Corollary}
\newtheorem{prop}[equation]{Proposition}
\theoremstyle{definition}
\newtheorem{ex}[equation]{Exercise}
\newtheorem{exm}[equation]{Example}
\newtheorem{defn}[equation]{Definition}
\newtheorem{claim}[equation]{Claim}
\newtheorem{conj}[equation]{Conjecture}
\newtheorem{ques}[equation]{Question}
\theoremstyle{remark}
\newtheorem{rem}[equation]{Remark}
\newtheorem*{fct}{Fact}
\newtheorem*{note}{Note}

% Crefnames, allowing me to reference multiple theorems at once
\crefname{thm}{Theorem}{Theorems}
\crefname{lem}{Lemma}{Lemmas}
\crefname{cor}{Corollary}{Corollaries}
\crefname{prop}{Proposition}{Propositions}
\crefname{ex}{Exercise}{Exercises}
\crefname{exm}{Example}{Examples}
\crefname{defn}{Definition}{Definitions}
\crefname{claim}{Claim}{Claims}
\crefname{conj}{Conjecture}{Conjectures}
\crefname{ques}{Question}{Questions}
\crefname{rem}{Remark}{Remarks}
\crefname{fct}{Fact}{Facts}
\crefname{note}{Note}{Notes}

\newcommand{\term}{\emph} % e.g. "The \term{trace} is defined to be..."

% other shortcuts I use for live-TeXing
\newcommand{\vp}{\varphi}
\newcommand{\e}{\varepsilon}
\newcommand{\inj}{\hookrightarrow}
\newcommand{\surj}{\twoheadrightarrow}
\newcommand{\id}{\mathrm{id}}
\newcommand{\pt}{\mathrm{pt}}
\newcommand{\many}[2][\dotsb]{#2 #1 #2} % optional argument for kind of dots
\newcommand{\bl}{\text{--}}
\newcommand{\TFAE}{The following are equivalent}
\newcommand{\TODO}{\textcolor{red}{TODO}}

% This allows \paren{...} to replace \left(...\right) (and similarly for \bkt). For
% \ang, \set, \abs, and \norm, I find myself using autoexpansion less often.
% This code, along with some other shortcuts, is duplicated in my problem set template;
% perhaps I should have them include a common file?
\DeclarePairedDelimiter\paren{(}{)}
\DeclarePairedDelimiter\ang{\langle}{\rangle}
\DeclarePairedDelimiter\abs{\lvert}{\rvert}
\DeclarePairedDelimiter\norm{\lVert}{\rVert}
\DeclarePairedDelimiter\bkt{[}{]}
\DeclarePairedDelimiter\set{\{}{\}}
\DeclarePairedDelimiter\bra{\langle}{\rvert}
\DeclarePairedDelimiter\ket{\lvert}{\rangle}
% Swap paren* and paren, etc., so that the normal version resizes by default.
% Meanwhile, one can use \paren*[\Big]{...} to customize the size easily.
\makeatletter
	\let\oldparen\paren
	\def\paren{\@ifstar{\oldparen}{\oldparen*}}
	\let\oldbkt\bkt
	\def\bkt{\@ifstar{\oldbkt}{\oldbkt*}}
\makeatother

% simplifying use of \DeclareMathOperator
\newcommand{\newoperator}[1]{\expandafter\DeclareMathOperator\csname #1\endcsname{\operatorname{#1}}}
\def\do#1{\newoperator{#1}}
\docsvlist{Aut,coker,Cotor,End,Ext,Hom,rank,sign,Tor,tr,Spec,Proj,QCoh}
\renewcommand{\Im}{\operatorname{Im}}
\DeclareMathOperator*{\colim}{colim}
\DeclareMathOperator*{\holim}{holim}
\DeclareMathOperator*{\hocolim}{hocolim}

\usepackage{microtype}
\usepackage{hyperref}
\hypersetup{
    colorlinks,
    linkcolor={red!50!black},
    citecolor={green!50!black},
    urlcolor={blue!80!black}
}
