The Adams spectral sequence is an important tool in homotopy theory. It converges to the stable homotopy groups of
the spheres (well, the most commonly studied variation), and so by conservation of effort, running the spectral
sequence must be very hard. Indeed,
\begin{itemize}
	\item determining the $E_2$-page is complicated, but is purely algebraic so can be aided with computer
	calculations, and
	\item determining the differentials is art and/or dark magic!
\end{itemize}
Though we understand a great deal of the structure of the Adams spectral sequence, including vanishing slopes and
infinite families of elements that survive to the $E_\infty$-page, Mahowald's uncertainty principle (not a real
theorem, at least not yet) states that any single method to determine Adams differentials must leave infinitely
many unresolved. Again, this is not a theorem, just an observation made after years of experience.

All this posturing aside, what does the Adams spectral sequence look like? Well, fix a prime $p$ and two spectra
$X$ and $Y$. Then the Adams spectral sequence has signature
\begin{equation}
\label{Adams_sig}
	E_2^{s,t} = \Ext_{\cA_p}^{s,t}(H^*(X;\F_p), H^*(Y; \F_p)) \Longrightarrow [Y, X]_p^\wedge.
\end{equation}
That is, it begins with Ext over the mod $p$ Steenrod algebra $\cA_p$, and converges to the $p$-completion of the
abelian group of homotopy classes of maps $Y\to X$. \TODO: convergence. Often, one takes $Y = \S$, so that this
calculates the $p$-completed homotopy groups of $X$. Most research has been done for the case $X = \S$; in this
case $p = 2$ is the hardest and most-studied; we understand the spectral sequence entirely up to about degree $t -
s = 63$, and blearily up to about degree $90$. (\TODO: many names go here.)
\begin{rem}
There are variants of the Adams spectral sequence where you begin with a spectrum $E$ and feed it the
$E$-cohomology of $X$ and $Y$, and work over the algebra of stable $E$-cohomology operations. Our case corresponded
to $E = H\F_p$. The case $E = \mathit{BP}$ is also common, in which case this is called the \term{Adams-Novikov
spectral sequence}; one also sees $E = \ko$. The $E_2$-page is not as nice as~\eqref{Adams_sig} in general, though.

Amusingly, you can take $E = H\Q$, and as soon as you know that the algebra of stable $\Q$-cohomology operations is
just $\Q$ in degree $0$, the $H\Q$-based Adams spectral sequence collapses, recovering a fact you may already know:
rational stable homotopy is rational stable homology.
\end{rem}
Remember the mod $2$ Steenrod algebra from Wednesday? It's neither finitely generated nor commutative, which is to
say it's a massive headache. So in our class today, focused as we are on getting our hands dirty, we will work with
a special case of the Adams spectral sequence which is simple enough to actually teach in a day, yet complicated
enough to both feature the structures and ideas present in a more general Adams spectral sequence calculation, and
also something that's actually interesting and useful to calculate.

(Change-of-rings theorem here \TODO)

The most common reason to invoke this is Stong's calculation that
\begin{equation}
	H^*(\ko;\F_2) \cong \cA\otimes_{\cA(1)} \F_2,
\end{equation}
where $\cA(1) \coloneqq \ang{\Sq^1, \Sq^2}$. So if you want to know the $2$-primary $\ko$-theory of something, the
change-of-rings theorem and Künneth formula together imply the Adams spectral sequence for $\ko\wedge X$ has
signature
\begin{equation}
	E_2^{s,t} = \Ext_\cA^{s,t}(\cA\otimes_{\cA(1)} H^*(X;\F_2), \F_2) \cong \Ext_{\cA(1)}^{s,t}(H^*(X,\F_2), \F_2).
\end{equation}
This is good, because $\cA(1)$ is \emph{much} smaller than $\cA$: it's eight-dimensional over $\F_2$. In practice,
running the Adams spectral sequence is often (if not always) completely tractable in these cases: determining Ext,
resolving differentials, and resolving extensions. Our goal today is to teach you how to do this --- not to
develop everything by hand, but to build on what's already written down to use this as a tool to solve whatever
probelms you might have in practice.
\begin{rem}
For the bordism-minded in the audience, at the prime $2$, $\MSpin$ splits as a sum of various connective covers of
$\ko$ and $H\F_2$s; each of these pieces can be tackled with this change-of-rings trick. Shearing arguments
generalize this to related kinds of bordism, e.g. spin\textsuperscript{$c$}, pin\textsuperscript{$+$}, or
pin\textsuperscript{$-$} bordism, and so these kinds of bordism of bounded-below spectra can be fed relatively
routinely to this method of calculation. It is ultimately for this reason that you sometimes see Adams spectral
sequence calculations in physics!
\end{rem}
There are several references for this stuff, and I recommend Beaudry and Campbell's paper ``A guide for computing
stable homotopy groups,'' which develops the theory from relatively few assumptions, and includes several example
calculations.

\subsection{Drawing $\cA(1)$-modules}



\subsection{Determining Ext}
% and also the H^*,*(A(1))-module structure/the Yoneda product!
Let $M$ be an $\cA(1)$-module that we care about. There are many ways to compute $\Ext(M)$.
\begin{enumerate}
	\item\label{look_it_up} Use someone else's preexisting computation (yes, really!).
	\item\label{min_res} Write down a minimal resolution of $M$.
	\item\label{SES_LES} Put $M$ into a short exact sequence of $\cA(1)$-modules and use the induced long exact
	sequence in Ext.
	\item\label{kernel_method} Notice that $M$ looks a lot like another $\cA(1)$-module whose Ext you already know.
	\item\label{apply_CoR} Clever applications of the change-of-rings formula.
\end{enumerate}
But before that, I'd like to discuss a little more structure that $\Ext(M)$ has: it is a module over the algebra
$H^{*,*}(\cA(1))\coloneqq \Ext(\F_2)$.

By definition, Ext is the derived functor of Hom, but elements of Ext can be concretely represented by extensions
of $\cA(1)$-modules. For $s > 0$, as a set, $\Ext^{s,t}(M, N)$ is the equivalence classes of long exact sequences
\begin{equation}
	\xymatrix{0\ar[r] & \Sigma^t N\ar[r] & P_s\ar[r] & P_{s-1}\ar[r] & \dots\ar[r] & P_1\ar[r] & M\ar[r] & 0,}
\end{equation}
where two such sequences are equivalent if there are maps between all these modules such that the maps on $M$ and
$N$ are the identity, and that the relevant big ol' diagram commutes. This has an abelian group structure using
something called \term{Baer sum}, but since $2 = 0$ it doesn't come up in practice much.

What \emph{does} come up is the \term{Yoneda product}. Given an extension $0\to \Sigma^tL\to\dotsb \to M\to 0$ of
length $s$ and an extension $0\to \Sigma^{t'}M\to \dotsb\to N\to 0$ of length $s'$, you can compose them to an
extension
\begin{equation}
	\xymatrix{
		0\ar[r] & \Sigma^{t+t'}L\ar[r] & \dotsb\ar[r] & \Sigma^t P'_{s'}\ar[dr]\ar[rr] && \dotsb P_1\ar[r] &
		\dotsb\ar[r] & N\\
		&&&& \Sigma^t M
	}
\end{equation}
and this defines a product
\begin{equation}
	\Ext_{\cA(1)}^{s,t}(L, M)\otimes \Ext_{\cA(1)}^{s',t'}(M, N)\longrightarrow \Ext_{\cA(1)}^{s+s', t+t'}(L, N).
\end{equation}
called the Yoneda product. When $L = M = N = \F_2$, this makes $\Ext_{\cA(1)}^{*,*}(\F_2, \F_2)$ into a bigraded
algebra, which is denoted $H^{*,*}(\cA(1))$, and in general this naturally makes $\Ext_{\cA(1)}^{*,*}(M, \F_2)$
into an $H^{*,*}(\cA(1))$-module. This structure is very useful in Adams spectral sequence calculations (and this
is true no matter whether you work over $\cA(1)$ or $\cA$ or any other subalgebra).

\begin{rem}[s = 0]
By general derived functor machinery, $\Ext^{0,t}(M, N) = \Hom(M, \Sigma^t N)$. The Yoneda product is \TODO.
\end{rem}

Let's dig into this algebra and module structure a bit. One can calculate that
\begin{equation}
	H^{*,*}(\cA(1)) \cong\F_2[h_0, h_1, v, w]/(h_0h_1, h_1^3, h_1v, v^2 = h_0^2w)
\end{equation}
with $\abs{h_0} = (1, 1)$, $\abs{h_1} = (1, 2)$, $\abs{v} = (3, 7)$, and $\abs w = (4, 12)$. This description,
though complete, lacks a certain --- you know, what does this \emph{look} like?

As this is the $E_2$-page of a spectral sequence (converging to the $\ko$-homology of a point), let's draw it like
one! One important point is that the Adams spectral sequence grading is $(t-s, s)$, so, e.g.\ $h_0$ will be in
bidegree $(0, 1)$, $h_1$ will be in bidegree $(1, 1)$, etc. Anyways, what you get is
\begin{sseqdata}[name=Extalgebra, classes=fill, scale=0.5, xrange={0}{10}, yrange={0}{6}]
\class(0, 0)\AdamsTower{}
\class(1, 1)\structline(0, 0)(1, 1)
\class(2, 2)\structline

\class(4, 3)\AdamsTower{}

\class(8, 4)\AdamsTower{}
\class(9, 5)\structline(8, 4)(9, 5)
\class(10, 6)\structline
\end{sseqdata}
\begin{equation}
\begin{gathered}
	\printpage[name=Extalgebra, page=2]
\end{gathered}
\end{equation}
\TODO: label $h_0$, $h_1$, $v$, and $w$.

In general, given an $\cA(1)$-module $M$, when drawing $\Ext_{\cA(1)}(M, \F_2)$, we will draw in the
$H^{*,*}(\cA(1))$-action as follows: $h_0$ by vertical lines, $h_1$ by diagonal lines. We won't draw the effects of
the remaining generators, but they're occasionally helpful to keep in mind.
\begin{rem}
If you said, a single dot corresponds to a $\F_2$ summand, maybe an infinite tower corresponds to a $\Z$ somehow,
then reading along the $x$-axis you get $\Z$, $\Z/2$, $\Z/2$, $0$, $\Z$, $0$, $0$, $0$ repeating. If this sounds
familiar, this is not a coincidence. We'll get there in a sec.
\end{rem}
Now, here are some examples of commonly arising $\cA(1)$-modules and their Exts. These and others you can find fit
into the ``look it up'' technique.
\begin{exm}
Of course, $\Ext_{\cA(1)}(\Sigma^t\cA(1), \F_2)\cong \Sigma^t\F_2$: a single dot in the Ext chart.
\end{exm}
\begin{exm}
The $\cA(1)$-module $\Sigma^{-2}\widetilde H^*(\CP^2)$, consisting of two points joined by a $\Sq^2$, is often
denoted $C\eta$\dots
\end{exm}
\begin{exm}
The joker $J$\dots
\end{exm}
\begin{exm}
The upside-down question mark (sometimes called the Spanish question mark) (\TODO: symbol)\dots
\end{exm}
\begin{exm}
This thing called $M_\infty$\dots, useful for studying $H^*(\RP^\infty)$ and its variants (stunted projective
spaces).
\end{exm}

Another way to compute Ext is using a minimal resolution.
\TODO: definition. Ideally add an example, but don't dwell on it during the lecture.


Another way to compute Ext is to use the fact that a short exact sequence of $\cA(1)$-modules induces a long exact
sequence in Ext. \TODO: example computing Ext of upside-down question mark (or, since active learning, let's use
the ordinary question mark?). This is the most flexible way to compute Ext in practice; specific other methods work
better on specific modules, but the LES is quite general. (\TODO: many more examples for the interested reader can
be found\dots)

Note: I'll leave this in the notes but not discuss it in the lecture, for time reasons.
\begin{exm}[Use a preexisting minimal resolution]
Modules which look similar should have similar Exts. Here's a concrete implementation of this, for the augmentation
ideal $I\coloneqq \textcolor{Green}{\Sigma^{-1}}\ker(\cA(1)\to\F_2)$. This is the remainder of the first step of
the minimal resolution $P^2\to P^1\to P^0\to\F_1$, so it has a minimal resolution beginning $P^2\to P^1\to I$. This
implies an $H^{*,*}(\cA(1))$-equivariant identification
\begin{equation}
	\Ext_{\cA(1)}^{s, t}(I, \F_2) \cong\Ext_{\cA(1)}^{s+1, \textcolor{Green}{t+1}}(\F_2, \F_2).
\end{equation}
There'll be a few exercises using similar tricks, but this isn't as crucial of a method to know.
\end{exm}
Again, this is appearing in the notes but will be skated over at best in the lecture.
\begin{exm}
Using the change-of-rings formula to compute the Ext of $C\eta$ very quickly. Again, elegant but not as flexible.
\end{exm}

% maybe I should use the example of D_{2n} that I like to use; maybe I should use S_4 because no Thom spectra
% floating around
\subsection{Running the spectral sequence: differentials}
% what do the differentials look like?
In the Adams grading, a $d_r$ differential moves one unit to the left and $r$ units upward. Typically, Adams
differentials are difficult, but we can take advantage of some structure.
\begin{enumerate}
	\item Differentials commute with the $H^{*,*}(\cA(1))$-action. This allows you to compute one differential and
	then move it up an $h_0$-tower, for example. Another crucial use is to infer that if $h_ix = 0$ but $h_iy\ne
	0$, then $d_r(x)\ne y$ for any $r$. For example, this method shows that all differentials in the Adams spectral
	sequence for $\ko$ (i.e., using $\Ext_{\cA(1)}(\F_2)$) vanish!
	\item There is a very useful theorem of Margolis which says that $\cA$-module summands in the $\F_2$-cohomology
	of a spectrum $X$ correspond precisely to summands of $H\F_2$ splitting off of $X$. The upshot is that in the
	$E_2$-page of the Adams spectral sequence for $\ko\wedge Y$, anything coming from an $\cA(1)$ summand in
	$H^*(Y;\F_2)$ neither admits nor receives nonzero differentials. These two methods are pretty powerful when
	used together.
	\item After this, one has to be crafty, and the standard tricks appear: use functoriality of the Adams spectral
	sequence to force differentials to vanish (or to not vanish); compute part of the answer using another spectral
	sequence, such as Atiyah-Hirzebruch, and use this to infer what differentials must vanish. There are ways to
	use Massey products and Toda brackets to infer differentials, but this is not something I understand; sorry
	about that.
\end{enumerate}

\subsection{Extension questions}
So you've finished running your Adams spectral sequence calculations and arrived at the $E_\infty$-page. How do you
address extensions? Here are the ways.
\begin{enumerate}
	\item The $H^{*,*}(\cA(1))$-action on the $E_\infty$-page lifts to the $\ko_*$-action on $\ko_*(X)$! To be
	precise, $h_0$ lifts to multiplication by $2$, $h_1$ lifts to the action by $\eta\in\ko_1$, and $w$ lifts to
	action by the periodicity element. This is the crucial first step to solving extension problems. For example,
	an infinite tower linked by $h_0$s lifts to the $2$-adics (which came from $\Z$ if your spectrum, before
	$2$-completion, was finite). If there are $k$ dots linked by $h_0$s, you've found a $\Z/2^k$.

	The converse is not true: there can be ``hidden extensions'' not visible to the $H^{*,*}(\cA(1))$-action.
	\item Margolis' theorem, the sequel: since your summands coming from $\cA(1)$s split off of $\ko\wedge X$, they
	cannot participate in hidden extensions.
	\item There are a few other tricks. One common one is that the relation $h_0h_1 = 0$ lifts to $2\eta = 0$. That
	means that in $\ko_*(X)$, if $x = 2y$, then $\eta x = 0$. Duh, you say, but this is useful for ruling out some
	hidden extensions. Mapping to/from other spectral sequences is also useful sometimes.
\end{enumerate}

\begin{rem}[Variants]
% tmf, ku, BP<2>/tmf_1(3)?
\end{rem}
