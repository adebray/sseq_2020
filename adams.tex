The Adams spectral sequence is an important tool in homotopy theory. It converges to the stable homotopy groups of
the spheres (well, the most commonly studied variation), and so by conservation of effort, running the spectral
sequence must be very hard. Indeed,
\begin{itemize}
	\item determining the $E_2$-page is complicated, but is purely algebraic so can be aided with computer
	calculations, and
	\item determining the differentials is art and/or dark magic!
\end{itemize}
Though we understand a great deal of the structure of the Adams spectral sequence, including vanishing slopes and
infinite families of elements that survive to the $E_\infty$-page, Mahowald's uncertainty principle (not a real
theorem, at least not yet) states that any single method to determine Adams differentials must leave infinitely
many unresolved. Again, this is not a theorem, just an observation made after years of experience.

All this posturing aside, what does the Adams spectral sequence look like? Well, fix a prime $p$ and two spectra
$X$ and $Y$. Then the Adams spectral sequence has signature
\begin{equation}
\label{Adams_sig}
	E_2^{s,t} = \Ext_{\cA_p}^{s,t}(H^*(X;\F_p), H^*(Y; \F_p)) \Longrightarrow [Y, X]_p^\wedge.
\end{equation}
That is, it begins with Ext over the mod $p$ Steenrod algebra $\cA_p$, and converges to the $p$-completion of the
abelian group of homotopy classes of maps $Y\to X$. \TODO: convergence. Often, one takes $Y = \Sph$, so that this
calculates the $p$-completed homotopy groups of $X$. Most research has been done for the case $X = \Sph$; in this
case $p = 2$ is the hardest and most-studied; we understand the spectral sequence entirely up to about degree $t -
s = 63$, and blearily up to about degree $90$. (\TODO: many names go here.)
\begin{rem}
There are variants of the Adams spectral sequence where you begin with a spectrum $E$ and feed it the
$E$-cohomology of $X$ and $Y$, and work over the algebra of stable $E$-cohomology operations. Our case corresponded
to $E = H\F_p$. The case $E = \mathit{BP}$ is also common, in which case this is called the \term{Adams-Novikov
spectral sequence}; one also sees $E = \ko$. The $E_2$-page is not as nice as~\eqref{Adams_sig} in general, though.

Amusingly, you can take $E = H\Q$, and as soon as you know that the algebra of stable $\Q$-cohomology operations is
just $\Q$ in degree $0$, the $H\Q$-based Adams spectral sequence collapses, recovering a fact you may already know:
rational stable homotopy is rational stable homology.
\end{rem}
Remember the mod $2$ Steenrod algebra from Wednesday? It's neither finitely generated nor commutative, which is to
say it's a massive headache. So in our class today, focused as we are on getting our hands dirty, we will work with
a special case of the Adams spectral sequence which is simple enough to actually teach in a day, yet complicated
enough to both feature the structures and ideas present in a more general Adams spectral sequence calculation, and
also something that's actually interesting and useful to calculate.

(Change-of-rings theorem here \TODO)

The most common reason to invoke this is Stong's calculation that
\begin{equation}
	H^*(\ko;\F_2) \cong \cA\otimes_{\cA(1)} \F_2,
\end{equation}
where $\cA(1) \coloneqq \ang{\Sq^1, \Sq^2}$. So if you want to know the $2$-primary $\ko$-theory of something, the
change-of-rings theorem and Künneth formula together imply the Adams spectral sequence for $\ko\wedge X$ has
signature
\begin{equation}
	E_2^{s,t} = \Ext_\cA^{s,t}(\cA\otimes_{\cA(1)} H^*(X;\F_2), \F_2) \cong \Ext_{\cA(1)}^{s,t}(H^*(X,\F_2), \F_2).
\end{equation}
This is good, because $\cA(1)$ is \emph{much} smaller than $\cA$: it's eight-dimensional over $\F_2$. In practice,
running the Adams spectral sequence is often (if not always) completely tractable in these cases: determining Ext,
resolving differentials, and resolving extensions. Our goal today is to teach you how to do this --- not to
develop everything by hand, but to build on what's already written down to use this as a tool to solve whatever
probelms you might have in practice.
\begin{rem}
For the bordism-minded in the audience, at the prime $2$, $\MTSpin$ splits as a sum of various connective covers of
$\ko$ and $H\F_2$s; each of these pieces can be tackled with this change-of-rings trick. Shearing arguments
generalize this to related kinds of bordism, e.g. \spinc, \pinp, or \pinm bordism, and so these kinds of bordism of
bounded-below spectra can be fed relatively routinely to this method of calculation. It is ultimately for this
reason that you sometimes see Adams spectral sequence calculations in physics!
\end{rem}
There are several references for this stuff, and I recommend Beaudry and Campbell's paper ``A guide for computing
stable homotopy groups,'' which develops the theory from relatively few assumptions, and includes several example
calculations.

\subsection{Drawing $\cA(1)$-modules}



\subsection{Determining Ext}
% and also the H^*,*(A(1))-module structure/the Yoneda product!


% maybe I should use the example of D_{2n} that I like to use; maybe I should use S_4 because no Thom spectra
% floating around
\subsection{Running the spectral sequence: differentials}
% what do the differentials look like?

% linearity

% Margolis' theorem


% other tricks


\subsection{Extension questions}
% using h0-linearity

% using Margolis' theorem


% 2\eta = 0

% other tricks



\begin{rem}[Variants]
% tmf, ku, BP<2>/tmf_1(3)?
\end{rem}
